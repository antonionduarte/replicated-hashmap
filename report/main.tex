\documentclass[sigconf]{acmart}

\usepackage{multirow}
\usepackage{graphicx}
\usepackage[ruled, vlined]{algorithm2e}
% \usepackage{algorithm}
% \usepackage{algpseudocode}

\graphicspath{ {images/} }

\SetKwBlock{Interface}{Interface:}{}
\SetKwBlock{State}{State:}{}
\SetKwBlock{Requests}{Requests:}{}
\SetKwBlock{Indications}{Indications:}{}

\SetKwProg{UponTimer}{Upon Timer}{ do:}{}
\SetKwProg{Procedure}{Procedure}{ do:}{}
\SetKwProg{Upon}{Upon}{ do:}{}
\SetKwProg{If}{If}{ do:}{}
\SetKwProg{Foreach}{Foreach}{ do:}{}

\SetKwComment{Comment}{/* }{ */}
\SetKw{Trigger}{Trigger}
\SetKw{Call}{Call}
\SetKw{CancelTimer}{Cancel Timer}
\SetKw{SetupTimer}{Setup Timer}
\SetKw{SetupPeriodicTimer}{Setup Periodic Timer}

%%
%% \BibTeX command to typeset BibTeX logo in the docs
\AtBeginDocument{%
  \providecommand\BibTeX{{%
    \normalfont B\kern-0.5em{\scshape i\kern-0.25em b}\kern-0.8em\TeX}}}

%% Rights management information.  This information is sent to you
%% when you complete the rights form.  These commands have SAMPLE
%% values in them; it is your responsibility as an author to replace
%% the commands and values with those provided to you when you
%% complete the rights form.
\setcopyright{acmcopyright}
\copyrightyear{2022}
\acmYear{2022}
%\acmDOI{10.1145/1122445.1122456}

%% These commands are for a PROCEEDINGS abstract or paper.
\acmConference[ASD22/23]{The first project delivery of ASD2223}{2022}{Faculdade de Ciências e Tecnologia, NOVA University of Lisbon, Portugal}
\acmBooktitle{The Projects of ASD - first delivery, 2021, Faculdade de Ciências e Tecnologia, NOVA University of Lisbon, Portugal}
%\acmPrice{15.00}
%\acmISBN{978-1-4503-XXXX-X/18/06}


%%
%% Submission ID.
%% Use this when submitting an article to a sponsored event. You'll
%% receive a unique submission ID from the organizers
%% of the event, and this ID should be used as the parameter to this command.
%%\acmSubmissionID{123-A56-BU3}

%%
%% The majority of ACM publications use numbered citations and
%% references.  The command \citestyle{authoryear} switches to the
%% "author year" style.
%%
%% If you are preparing content for an event
%% sponsored by ACM SIGGRAPH, you must use the "author year" style of
%% citations and references.
%% Uncommenting
%% the next command will enable that style.
%%\citestyle{acmauthoryear}

%%
%% end of the preamble, start of the body of the document source.
\begin{document}

%%
%% The "title" command has an optional parameter,
%% allowing the author to define a "short title" to be used in page headers.
\title{Strongly Consistent Replicated HashMap}

%%
%% The "author" command and its associated commands are used to define
%% the authors and their affiliations.
%% Of note is the shared affiliation of the first two authors, and the
%% "authornote" and "authornotemark" commands
%% used to denote shared contribution to the research.
\author{António Duarte}
\authornote{Student number 58278. %responsibility?
}
\email{an.duarte@campus.fct.unl.pt}
\affiliation{%
    \institution{MIEI, DI, FCT, UNL}
}

\author{Diogo Almeida}
\authornote{Student number 58369. %responsibility?
}
\email{daro.almeida@campus.fct.unl.pt}
\affiliation{%
    \institution{MIEI, DI, FCT, UNL}
}

\author{Diogo Fona}
\authornote{Student number 57940. %responsibility?
}
\email{d.fona@campus.fct.unl.pt}
\affiliation{%
    \institution{MIEI, DI, FCT, UNL}
}

%%
%% By default, the full list of authors will be used in the page
%% headers. Often, this list is too long, and will overlap
%% other information printed in the page headers. This command allows
%% the author to define a more concise list
%% of authors' names for this purpose.
\renewcommand{\shortauthors}{Duarte, Almeida, and Fona.}

%%
%% The abstract is a short summary of the work to be presented in the
%% article.
\begin{abstract}

\end{abstract}

%% This command processes the author and affiliation and title
%% information and builds the first part of the formatted document.
\maketitle

\section{Introduction}

A replicated hash map is a distributed data structure that allows for efficient access to data across a network of machines. It uses the concept of a hash map, which is a data structure that stores key-value pairs and allows for efficient lookup of values based on their associated keys. In a replicated hash map, multiple copies of the hash map are maintained on different machines in the network, and agreement protocols can be used to ensure that all copies are kept consistent with each other.

State machine replication is a technique used in distributed systems to ensure that the same sequence of operations is performed on all replicas of a given state machine. This is typically achieved through the use of an agreement protocol, such as Paxos or Raft, which is used to coordinate the actions of the replicas and reach consensus on the order in which operations should be applied.

Paxos \cite{leslie1998part} \cite{lamport2001paxos} \cite{van2015paxos} and Multi-Paxos \cite{lamport2001paxos} \cite{du2009multi} \cite{van2015paxos} are examples of agreement protocols that are commonly used in distributed systems to achieve consistency in replicated data structures. The Paxos algorithm was first described by Leslie Lamport in 1998 and is a method for reaching consensus among a group of nodes in a distributed system. Multi-Paxos is a variant of the Paxos algorithm that allows for improved performance in scenarios where there are multiple replicated data structures, or multiple instances of the Paxos algorithm running concurrently.

In both Paxos and Multi-Paxos, nodes in the distributed system communicate with each other using a series of rounds, during which they propose values, vote on the proposed values, and ultimately agree on a single value to be chosen. Once consensus has been reached, the agreed-upon value can be safely used by the nodes in the system, ensuring that all copies of the replicated data are consistent with each other.

The remainder of this article is organized as follows: In Section 2 we will be going over the Related Work, explaining more thoroughly the mentioned protocols; In Section 3 we will go over the details of our implementation of the protocols and how they interact with each other; In Section 4 we present the results of our experimental evaluation, testing the system with an increasing throughput of clients issuing operations and membership changes. In Section 5 we will summarize and take conclusions on this work.
\section{Related Work}

\section{Implementation}

% I kept this pseudocode example here.
\begin{algorithm}
    % TODO: Write the Pseudocode to how the PubSub interacts with PlumTree
    \Interface{
        \Requests{
            Subscribe(topic)\;
            Unsubscribe(topic)\;
            Publish(msg, topic)\;
        }

        \Indications{
            PubsubDeliver(msg)\;
            SubscriptionReply(topic)\;
            UnsubscriptionReply(topic)\;
        }
    }

    \texttt{\\}

    \State{
        $seenMessages$\;
        $subscribedTopics$\;
    }

    \texttt{\\}

		\Upon{Init}{
			$seenMessages \leftarrow \{\}$\;
			$subscribedTopics \leftarrow \{\}$\; 
		}

		\texttt{\\}

    \Upon{PlumtreeDeliver(sender, \{\textbf{GOSSIP}, msg, topic)\}} {
        \If{$topic \in subscribedTopics \land msg \notin seenMessages$} {
            $seenMessages \leftarrow seenMessages \cup \{message\}$\;
            \Trigger PubsubDeliver(msg)\;
        }
    }

    \texttt{\\}

    \Upon{Subscribe(topic)} {
        $subscribedTopics \leftarrow subscribedTopics \cup \{topic\}$\;
        \Trigger{SubscriptionReply(topic)}\;
    }

    \texttt{\\}

    \Upon{Unsubscribe(topic)} {
        $subscribedTopics \leftarrow subscribedTopics \setminus \{topic\}$\;
        \Trigger{UnsubscriptionReply(topic)}\;
    }

    \texttt{\\}

    \Upon{Publish(\{message, topic\})} {
        \Trigger PlumtreeBroadcast(\{message, topic\})\;
        
        \If{$topic \in subscribedTopics$} {
            \Trigger PubsubDeliver(msg)\;
        }
    }
    \caption{Unstructured Publish-Subscribe}
		\label{alg:uns_pubsub}
\end{algorithm}

% Maybe write the PseudoCode to how HyParView interacts with PlumTree

\section{Experimental Evaluation}

\section{Conclusions}

\subsection{Future Work}

%%
%% The acknowledgments section is defined using the "acks" environment
%% (and NOT an unnumbered section). This ensures the proper
%% identification of the section in the article metadata, and the
%% consistent spelling of the heading.
\begin{acks}
    To João Leitão, for teaching us how to implement protocols and write good reports.
\end{acks}

%%
%% The next two lines define the bibliography style to be used, and
%% the bibliography file.
\bibliographystyle{ACM-Reference-Format}
\bibliography{acmart}

\end{document}
\endinput
%%
%% End of file `sample-sigconf.tex'